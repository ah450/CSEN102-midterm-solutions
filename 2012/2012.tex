\documentclass[11pt,a4paper]{article}
\usepackage{listings}
\usepackage{color}
\usepackage{enumitem}
\usepackage{mathtools}
\begin{document}
\definecolor{orange}{rgb}{1,0.5,0}
\definecolor{darkgreen}{rgb}{0,0.6,0}
\lstset{
language=Python,
numbers=none,
breaklines=true,
breakatwhitespace=false,
frame=single,
keepspaces=true,
columns=flexible,
tabsize=2,
showspaces=false,
showstringspaces=false,
rulecolor=\color{black},
keywordstyle=\color{blue},
commentstyle=\color{darkgreen},
basicstyle=\footnotesize,
stringstyle=\color{orange}
}
\title{CSEN 102 - Midterm 2012 solutions}
\maketitle

\section{Conditional}
Your task is to implement an algorithm that can calculate your maximum heart rate and your optimal
training pulse for fat-burning, endurance increase or cardiovascular system improvement.
The Formula for calculating the maximum heart rate (Pulse) depends on the age and the gender:
\begin{itemize}
  \item For men: 200 - Age
  \item For women 226 - Age
\end{itemize}

The training can be classified into the following zones

\begin{description}
  \item[Health zone:] This amounts to 50-60\& of the maximum heart rate.
  Within this pulse range particularly the cardiovascular system will be invigorated. This range is
  particularly suitable for beginners.
  \item[Fat burning zone:] This amounts to 60-70\% of the maximum heart rate.
Within this pulse range, most calories from fat are burned. Furthermore the cardiovascular system
will be trained.
\item[Anaerobic zone:] This amounts to 80-90\% of the maximum heart rate.
Within this pulse range, the body cannot cover the oxygen demand any longer. This range is for
the development of power and muscle mass.

\item[Red zone:] This amounts to 90-100\% of the maximum heart rate.
This pulse range should be handled with caution. It is dangerous for beginners and can be harmful
for the heart.
\end{description}

\begin{enumerate}[label=(\alph*)]
  \item Write an algorithm that given the age and the gender should display the different zones. For example,
your algorithm should display the following for a 45 years old man:

\begin{tabular}{l r}
  Health zone: & Between 88 and 105 \\
  Fat-burning zone: & to 122 \\
  Aerobic zoe: & to 140 \\
  Anaerobic zone: & to 158 \\
  Maximum heart rate / Red zone: & to 175 \\
\end{tabular}

\begin{description}
  \item[Solution]  \hfill \\
  \lstinputlisting{sol_1_a.py}
\end{description}

\newpage
\item Write an algorithm that given the age, the gender and the heart rate will display the corresponding
zone. For example, the algorithm should display for a 45 years old man and heart rate of 190, the
following message
\begin{verbatim}
Red Zone
\end{verbatim}
Your algorithm should consist of only \emph{if} statements, i.e. you are not allowed to use any \emph{else}
statements.

\begin{description}
  \item[Solution]  \hfill \\
  \lstinputlisting{sol_1_b.py}
\end{description}
\end{enumerate}

\newpage

\section{Sequential (a) + Iteration (b)}

Many people begin running because they want to lose weight. As one of the most vigorous exercises
out there, running is an extremely efficient way to burn calories and drop pounds. However, there
are other ways to burn calories. In the following several activities are listed to burn around 250
calories:

\begin{enumerate}[label=(\alph*)]
  \item
\begin{description}
  \item[Dancing:] 52 minutes
  \item[Doing yoga:] 90 minutes
  \item[Riding a bike:] 30 minutes
  \item[Rollerblading:] 18 minutes
  \item[Running:] 23 minutes
  \item[Standing while talking on the phone:] 120 minutes
  \item[Horseback riding:] 57 minutes
\end{description}

So as you can see, what will only take a few minutes to consume, could take a large amount of time
to burn off in the gym.
Assume that burning calories is proportional to the time needed in an activity (which is in general
not the case), write an algorithm that given the calories will output the activities above with the
time needed to burn those calories.
\lstinputlisting{sol_2_a.py}
\newpage
 \item A lady decides to lose weight to be prepared for her wedding. Assume that she is able to loose 5\%
from her weight every month by doing a strict diet.

\begin{enumerate}
  \item Write an algorithm that given the weight of a lady and her target weight will calculate how
many months the diet will last.
\begin{description}
  \item[Solution] \hfill \\
  \lstinputlisting{sol_2_b_1.py}
\end{description}
\end{enumerate}
\item Write an algorithm that given the weight, the current date and the date of the wedding will
calculate the weight of the lady during her wedding day. The date is given by day, month and
year.
Your algorithm should display the number of days left to the wedding as well as the weight.
For simplicity take the integer division of the days left by 30 to get the number of months.
\begin{description}
  \item[Solution] \hfill \\
   \lstinputlisting{sol_2_b_2.py}
\end{description}
\end{enumerate}

\newpage

\section{Tracing}
Given the following Python code fragment:
\lstinputlisting[frame=none]{example_3.py}
\begin{enumerate}[label=(\alph*)]
  \item  What is the output of the code above for a=1, b=2 and c=6. Use a tracing table to trace the while
loop.
\begin{description}
  \item[Solution] \hfill \\
  \begin{tabular}{| c | c | c | c| }
    \hline
    a & b & c & Output so far \\ \hline
    1 & 2 & 6 & \\ \hline
    1 & 4 & 3 & 1,4,3 \\ \hline
    4 & 15 & 1 & 1,4,3,4,16,1 \\ \hline
    64 & 256 & 0 & 1,4,3,16,1,64,256,0 \\ \hline
   \end{tabular}
\end{description}
\item  What is the value of the b after the execution of the code for any values of a, b and c.
\begin{description}
  \item[Solution] \hfill \\
  The value of the \emph{b} is $ b^{2^i} $ where $ i $ is the number of iterations of the while loop.
  Since the \emph{c} is divided by 2 in every iteration, the \emph{i} will be equal to $ ceil(\log_2 c) $. Therefore,
  the value of the \emph{b} after the execution of the code will be $ b^{2^{ceil(\log_2 c)}} $ (Thanks to Dr. Haythem O. Ismail for his help).
\end{description}
\end{enumerate}

\newpage

\section{Iteration over Lists}
\begin{enumerate}[label=(\alph*)]
  \item Given a list of integers, write an algorithm that checks whether a list is ordered in ascending order
or not.
For example for the list consisting of
\begin{verbatim}
5 4 12 16 1
\end{verbatim}
the algorithm should display
\begin{verbatim}
The list is not sorted
\end{verbatim}
For the list
\begin{verbatim}
5 10 12 16
\end{verbatim}

the algorithm should display
\begin{verbatim}
The list is sorted
\end{verbatim}
\textbf{Note:} For the case where the list is not sorted, your algorithm should stop right away. For the
example above, your algorithm should stop after comparing the 5 with the 4.
\begin{description}
  \item[Solution] \hfill \\
   \lstinputlisting{sol_4_a.py}
\end{description}
\newpage
\item Write an algorithm that uses the following approach to sort a list of positive integers excluding
zero. Each integer x of the input list will be stored in the index that corresponds to the value of x
in the output list.
For example, for the following list
\begin{center}
  \begin{tabular}{|c|c|c|c|}
    \hline
    4 & 3 & 7 & 2 \\
    \hline
  \end{tabular}
\end{center}
the algorithm should sort the elements in the following list:
\begin{center}
  \begin{tabular}{|c|c|c|c|c|c|c|}
    \hline
    -1 & 2 &3 & 4 & -1 & -1 & 7 \\
    \hline
  \end{tabular}
\end{center}
and displays the following message
\begin{verbatim}
2 3 4 7
\end{verbatim}
Please note that -1 stored in an index i means that the element with the value i does not exist in
the orginal list. Therefore, you are required to fill these cells with -1.
\newpage
\begin{description}
  \item[Solution] \hfill \\
  \lstinputlisting{sol_4_b.py}
\end{description}
\newpage
\item What is the drawback of the sorting algorithm (one English statement)?
\begin{description}
  \item[Solution] \hfill \\
  The drawback of this algorithm is the inefficeincy in terms of memory storage. For example, if the
list contains only one number: $ 10^6 $ the list will have $ 10^6 $ cells all filled with $ -1 $
except for the last cell.
\end{description}
\end{enumerate}

\end{document}